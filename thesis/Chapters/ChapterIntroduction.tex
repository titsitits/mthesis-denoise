\chapter{Introduction}\label{chap:Introduction}

%begoldintro
% what is image noise
%\section{Image noise}
Photographic image noise occurs as a camera sensor's ISO sensitivity increases to capture an image faster than it would in ideal conditions (``base ISO" sensitivity).\footnote{We often make references to ISO noise because increased ISO sensitivity is the main cause of noise, but it should be noted that there are other factors affecting the magnitude of noise acquired by the image sensor.} A fast shutter speed is often necessary even though there is insufficient light, for instance with handheld photography where a slow shutter speed results in blur caused by the camera shake, or when a dynamic subject results in motion blur. Increasing the ISO setting is akin to linearly amplifying the value measured on each sensor cell. A small initial value that is amplified is less accurate and more prone to errors; this amplified value in turn makes up photographic noise. 

%\section{Denoising methods}
% how are images denoised
Denoising is typically seen as the inverse problem of recovering the latent clean image from its noisy observation \cite{rednet}. 
% ADD traditional methods

Photographic development usually incorporates a conventional denoising method such as non-local means \cite{nlm} or wavelets-based methods \cite{wavelets-denoising}, or a combination thereof, sometimes making use of prior information about a specific sensor's response to different ISO noise values \cite{darktable-denoising}.

% why don't I need to cite bm3d here?
\acs{BM3D} \cite{bm3d} is a more powerful denoising method which has been used to benchmark newer techniques \cite{sidd}\cite{renoir}\cite{nind-ntire}\cite{learningtoseeinthedark}\cite{noise2noise}\cite{rednet}\cite{darmstadt}\cite{dncnn}\cite{microscopynoise}\cite{lossescomp}, though it is seldom used in off-the-shelf software (presumably because of its high computational cost and the need to specify a $\sigma$ noise value parameter). 

The use of deep learning to solve the denoising problem by directly generating the latent clean image, or in some cases recreating the noise and subtracting it from the observed image \cite{dncnn}, has been an active research area. However, while the synthetic results show state-of-the-art performance, testing on real data indicates that neural network-based solutions do not exceed the performance offered by \ac{BM3D} \cite{darmstadt}. It appears that neural networks simply learn the applied noise distribution and that ISO noise may involve additional transformations such as color distortions and loss of detail.

Some specialized work has shown that neural networks obtain state-of-the-art performance when trained with real data \cite{learningtoseeinthedark}\cite{microscopynoise}. We sought to assess the potential of deep learning applied to the denoising problem by expanding on this previous work through a dataset of images produced with various levels of ISO noise. This dataset can be used to train neural network models for general purpose denoising of high quality images.

Our work introduces an open dataset of DSLR-like\footnote{\label{largesensornote}We define a DSLR-like camera as one produced with an APS-C (25.1x16.7 mm) or larger sensor such as those present in most \acs{DSLR} and mirrorless cameras.} image sets with various levels of real noise caused by the digital sensor's increased ISO sensitivity. The dataset is large enough to be used for training and varies in content in order to model a great variety of scenes. Each scene was captured at multiple noise levels, with an average of 6 images per set, such that a model may be trained for blind denoising on the base ISO as well as beyond the highest ISO value of the camera by feeding it crops that have a random noise value.

The images in a set are all pixel-aligned. Some of the scenes include multiple ground-truth images which may be sampled at random during training; these would prevent the model from learning to reconstruct the random noise it has seen on one ground-truth, thus making it more difficult to overfit the noise. Overexposed areas are avoided in the ground-truth images because details are lost when the sensor is saturated and this could potentially give an advantage to higher ISO images in which the sensor is not necessarily saturated (notably in ISO-invariant cameras and high ISO pictures that are brightened using software).

The dataset is published in sRGB format on Wikimedia Commons, which is an open-platform that promotes continuous discussion and contribution.

%ADDED

We trained several convolutional neural network architectures presented in the literature, namely DnCNN \cite{dncnn}, \ac{RED-Net}, and U-Net \cite{unet}, and we analyzed different methods such as DnCNN \cite{dncnn}'s proposed residual learning. These models and methods show state-of-the-art results when presented with synthetic noise, but their performance on real photographic noise removal remained untested until now due to the lack of an appropriate dataset.
 % test data and architectures

In most tests a U-Net model was trained with the \ac{NIND} and validated over increasing ISO values; test sets were taken with the same Fujifilm X-T1, as well as a Canon EOS 500D DSLR camera to assess the generalization. We also attempted to combine other datasets with our training data to assess the potential loss in performance caused by generalization. %Do I need to list results here?% %We found good generalization which did not suffer a significant performance hit when compared to specialized models.
Finally, we investigated the use of \aclp{GAN} \cite{pix2pix} which may provide a solution to specific issues such as unnatural face smoothing and unmatched data.

We expect the \acl{NIND} will be useful for research and applications in image noise removal, because it provides the data needed for neural networks to outperform conventional methods. Our trained models' performance consistently exceed that of \ac{BM3D} and of models trained with artificial noise. The results we provide with generative neural networks provide a solution that is usable immediately, and the functional \acl{GAN} methods we explored provide groundwork for further work and application-specific use-cases.
% thesis statement: we expect the \acl{NIND} and the validated methods to prove useful for 

%endoldintro

%what/why/how (short)

%~4-5 (up to 10) typical
%\section{Image noise}
%\section{Current methods}
%\section{Our work}
